{# This is a Jinja2 template for a latex file.  Both use braces heavily! 
    We use the raw ... endraw statements to surround untouched latex 
    and we have to escape literal braces.
#}



{{ [0,1,2,3,4,5,6,7,8,9]|random }}





{#-
  Determine which versions should have separate solution files.
  If the cookiecutter context provides `versions_with_solutions`, use that
  CSV; otherwise default to all versions (when versions is set).
-#}







%<*driver>
%%!TEX TS-program = dtxmake
%%!TEX dtxmake-subengine = xelatexmk
\input docstrip.tex
\askforoverwritefalse
\generate{


  \file{\jobname-{{v}}.qns.tex}{\from{\jobname.dtx}{questions,{{v}}}}
  
  \file{\jobname-{{v}}.sol.tex}{\from{\jobname.dtx}{questions,solutions,{{v}}}}
  


    \file{\jobname.qns.tex}{\from{\jobname.dtx}{questions}}
    \file{\jobname.sol.tex}{\from{\jobname.dtx}{questions,solutions}}

}
\endbatchfile
%</driver>
%<*questions>
\documentclass[

,nyufonts   

%<solutions>,answers
]{{ "{" }}nyuquiz{{ "}" }}
\ProvidesFile
%<*dtx>
{{ (cookiecutter.exam_code + ".dtx") | embrace }}
%</dtx>


%<{{v}}&questions&!solutions>{{ (cookiecutter.exam_code + "-" + v + ".qns.tex") | embrace }}

%<{{v}}&questions&solutions>{{ (cookiecutter.exam_code + "-" + v + ".sol.tex") | embrace }}



%<questions&!solutions>{{ (cookiecutter.exam_code + ".qns.tex") | embrace }}
%<questions&solutions>{{ (cookiecutter.exam_code + ".sol.tex") | embrace }}

    [{% now 'local', '%Y/%m/%d' %} v0.0 {{cookiecutter.course_name}}, {{cookiecutter.term_name}}, {{cookiecutter.exam_name}}]
\title{{cookiecutter.exam_name | embrace }}
\author{{ cookiecutter.course_name | embrace }}
\date{{ cookiecutter.exam_date | localize_date | embrace }}
% Random seed(s) generated {% now 'local', '%c' %} 


%<{{v}}|>\def\randomseed{{ random_int() | embrace }}


\def\randomseed{{ random_int() | embrace }}


\pgfmathsetseed{\randomseed}
\usepackage[overload]{exam-randomizechoices}
\setrandomizerseed{\randomseed}
\ifluatex
  \directlua{math.randomseed(token.get_macro("randomseed"))}
  \usepackage{luacode}
  % See the luacode manual to understand why we need the \luaexec command here.
  % Should be single backlash, single percent, in double quotes
  \luaexec{PERCENT_CHAR = "\%"}
  \directlua{utils = require("utils")}
\fi





\usepackage{{ pkg | trim | embrace }}





\usetikzlibrary{{ lib | trim | embrace }}



\newcommand{\quizduration}{{ cookiecutter.quiz_duration | embrace }}

\newcommand{\quizduration}{20}

    



\begin{document}
\maketitle
\noindent This paper will be scanned and read by optical mark reading software.
Indicate your selections by filling in the circles \textbf{completely}.
If you wish to change your selection, erase your marks completely.
\bigskip

\insertidfields

\iflargeprint
  \begin{instructions}
    This is a \quizduration-minute quiz.  
    There are problems on the front and on the back.
    Notes and calculators are not allowed. Any student taking 
    this ``large print'' quiz version is allowed unlimited scrap paper.
    Please submit all scrap paper when submitting the quiz.
  \end{instructions}
\cleardoublepage
\else
  \begin{instructions}
    This is a \quizduration-minute quiz.  
    There are problems on the front and on the back.
    Notes, calculators, and scrap paper are not allowed.
  \end{instructions}
  \bigskip
  \bigskip
\fi 


\begin{questions}
\begin{question}[2]\label{MC-final}
    On which of these dates is the final exam?
    \begin{choices}
        \CorrectChoice{Tuesday, December 18}% correct
        \choice{Wednesday, December 19}% calculus exams
        \choice{Monday, December 17}% review session
        \choice{Thursday, December 20}% commencement
        \choice{Friday, December 14}% last day of class
    \end{choices}
\end{question}

\vspace{\stretch{1}}

\begin{question}[2]\label{MC-cheating}
    Which of these are examples of academic dishonesty?
    \begin{checkboxes}
        \CorrectChoice{Copying the homework of another student without collaboration and citation}
        \CorrectChoice{Falsifying documents to justify makeup exams}
        \CorrectChoice{Placing an ad on craiglist seeking someone to take your exam}
        \CorrectChoice{Asking homework questions on Yahoo! Answers and using answers found there as your own work}
    \end{checkboxes}
\end{question}

\vspace{\stretch{1}}

\clearpage

\begin{question}[3]\label{FR-why}
    Why are you taking this course?  What would you like to learn?
    \begin{solution}[\stretch{1}]        
    \end{solution}
\end{question}

\begin{question}[3]\label{FR-map}
    Draw a map of your home town, and indicate on the map where you live.
    \begin{solution}[\stretch{1}]        
    \end{solution}    
\end{question}

\end{questions}
\end{document}
   
%</questions>
